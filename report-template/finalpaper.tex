\documentclass[runningheads,a4paper]{llncs}

\usepackage[american]{babel}

\usepackage{graphicx}

%extended enumerate, such as \begin{compactenum}
\usepackage{paralist}

%put figures inside a text
%\usepackage{picins}
%use
%\piccaptioninside
%\piccaption{...}
%\parpic[r]{\includegraphics ...}
%Text...

%Sorts the citations in the brackets
%\usepackage{cite}

%for easy quotations: \enquote{text}
\usepackage{csquotes}

\usepackage[T1]{fontenc}

%enable margin kerning
\usepackage{microtype}

%better font, similar to the default springer font
%cfr-lm is preferred over lmodern. Reasoning at http://tex.stackexchange.com/a/247543/9075
\usepackage[%
rm={oldstyle=false,proportional=true},%
sf={oldstyle=false,proportional=true},%
tt={oldstyle=false,proportional=true,variable=true},%
qt=false%
]{cfr-lm}
%
%if more space is needed, exchange cfr-lm by mathptmx
%\usepackage{mathptmx}

%for demonstration purposes only
\usepackage[math]{blindtext}

%enable hyperref without colors and without bookmarks
\usepackage[
%pdfauthor={},
%pdfsubject={},
%pdftitle={},
%pdfkeywords={},
bookmarks=false,
breaklinks=true,
colorlinks=true,
linkcolor=black,
citecolor=black,
urlcolor=black,
%pdfstartpage=19,
pdfpagelayout=SinglePage
]{hyperref}
%enables correct jumping to figures when referencing
\usepackage[all]{hypcap}

%enable \cref{...} and \Cref{...} instead of \ref: Type of reference included in the link
\usepackage[capitalise,nameinlink]{cleveref}
%Nice formats for \cref
\crefname{section}{Sect.}{Sect.}
\Crefname{section}{Section}{Sections}
\crefname{figure}{Fig.}{Fig.}
\Crefname{figure}{Figure}{Figures}

\usepackage{xspace}
%\newcommand{\eg}{e.\,g.\xspace}
%\newcommand{\ie}{i.\,e.\xspace}
\newcommand{\eg}{e.\,g.,\ }
\newcommand{\ie}{i.\,e.,\ }

%introduce \powerset - hint by http://matheplanet.com/matheplanet/nuke/html/viewtopic.php?topic=136492&post_id=997377
\DeclareFontFamily{U}{MnSymbolC}{}
\DeclareSymbolFont{MnSyC}{U}{MnSymbolC}{m}{n}
\DeclareFontShape{U}{MnSymbolC}{m}{n}{
    <-6>  MnSymbolC5
   <6-7>  MnSymbolC6
   <7-8>  MnSymbolC7
   <8-9>  MnSymbolC8
   <9-10> MnSymbolC9
  <10-12> MnSymbolC10
  <12->   MnSymbolC12%
}{}
\DeclareMathSymbol{\powerset}{\mathord}{MnSyC}{180}

%improve wrapping of URLs - hint by http://tex.stackexchange.com/a/10419/9075
\makeatletter
\g@addto@macro{\UrlBreaks}{\UrlOrds}
\makeatother

% correct bad hyphenation here
\hyphenation{op-tical net-works semi-conduc-tor}

\begin{document}

%Works on MiKTeX only
%hint by http://goemonx.blogspot.de/2012/01/pdflatex-ligaturen-und-copynpaste.html
%also http://tex.stackexchange.com/questions/4397/make-ligatures-in-linux-libertine-copyable-and-searchable
%This allows a copy'n'paste of the text from the paper
\input glyphtounicode.tex
\pdfgentounicode=1

\title{Building a Bloodbank Database}
%If Title is too long, use \titlerunning
%\titlerunning{Short Title}

%Single insitute
\author{Nicholas Gordon}
%If there are too many authors, use \authorrunning
%\authorrunning{First Author et al.}
\institute{University of Memphis}

%Multiple insitutes
%Currently disabled
%
\iffalse
%Multiple institutes are typeset as follows:
\author{Firstname Lastname\inst{1} \and Firstname Lastname\inst{2} }
%If there are too many authors, use \authorrunning
%\authorrunning{First Author et al.}

\institute{
Insitute 1\\
\email{...}\and
Insitute 2\\
\email{...}
}
\fi
			
\maketitle

\begin{abstract}
  A report on the appropriateness, development, and construction of a
  database suitable for use by a blood bank. Includes short
  introduction to databases and competing vendors and web technologies
  as well. Discussed are the design decisions made, including a
  discussion of the schema used for the database, as well as the user
  interface design. Finally, a discussion of the implementation itself
  is given, including problems faced, analysis of the solution, and
  future expansions of the solution.
\end{abstract}

%% \keywords{...}

%%%%%%%%%%%%%%%%%%%%%%%%%%%%%%%%%%%%%%%%%%%%%%%%%%%%%%%%%%%%%%%%%%%%%%%%%%%%%%%
\section{Introduction}\label{sec:intro}
%%%%%%%%%%%%%%%%%%%%%%%%%%%%%%%%%%%%%%%%%%%%%%%%%%%%%%%%%%%%%%%%%%%%%%%%%%%%%%%
Databases are ubiquitous in computers. They are an extremely powerful
tool for organizing data, which is an imperative in computer science;
randomness is tantamount to uselessness. Once data is organized, it
can be manipulated in powerful ways to derive meaning from data. One
useful example of this is a blood bank. In a blood bank, the data are
a pool of patients, each with varying blood types, blood-borne
diseases, names, addresses, etc. Without organization, the best you
could hope to do is keep track of everyone. With organization, you can
generate lists of patients to contact for donation, who can donate to
some arbitrary patient, or even location-based demographics on
frequency of donation, disease distribution, etc.

Now we come to the database itself. We could simply organize these
lists by hand, periodically generating all of these metrics, but we
can use computers to muscle this data into formats we perfer for
us. This is the impetus for this project: generate a database that can
be populated with data from a blood bank which will provide to the
bank useful meaning from their data. The rest of this report will deal
with some knowledge upon which this report is incumbent, such as
differing database vendors and web technologies available for
interfacing with these databases, the overall design of the database
that has been produced, including the structure of the tables and the
database itself, and a discussion of the problems faced, the solution
presented, and its appropriateness to the problem at hand.

%%%%%%%%%%%%%%%%%%%%%%%%%%%%%%%%%%%%%%%%%%%%%%%%%%%%%%%%%%%%%%%%%%%%
\section{Related Work}\label{sec:related}
%%%%%%%%%%%%%%%%%%%%%%%%%%%%%%%%%%%%%%%%%%%%%%%%%%%%%%%%%%%%%%%%%%%%
In order to fully understand the decisions made during this project
and the design of the database given, some background knowledge needs
to be established. Given that a not-narrow audience is likely to be
reading a paper such as this, this explanation is warranted. Databases
often are just referred to as ``the database'' because of their
ubiquity, but the truth of the matter is that there are a great many
databases that any sysadmin must choose from, each having their own
quirks, costs, and intended uses. According to db-engines.com, which
uses a variety of metrics to determine database popularity, the five
most popular databases in February 2016\cite{db-engine} are, in order:
\begin{enumerate}
\item Oracle
\item MySQL
\item Microsoft SQL Server
\item MongoDB
\item PostgreSQL
\end{enumerate}
This list is not linear, however. The top three entries have
reasonably close numbers in terms of popularity, but MongoDB and
PostgreSQL have only about 30\% each of the share of Microsoft SQL
Server. Most databases are relational, which means their records are
stored in tables or similar, and relate multiple fields together in
items called records. These databases are the dominant paradigm, and
only MongoDB on this list is not a relational database. MongoDB fits
into another paradigm known as document-store database, which
effectively are a dictionary that stores complex objects in a
key-value system. The differences between these databases largely come
down to individual feature support, cost of use, and
enterprise-support. Of note is that MySQL, MongoDB, and PostgreSQL are
all open-source software and as such are free to use or have a free
version available.

Racket is a Scheme derivative, which emphasizes being a platform for
extending the language and being a full-spectrum programming language
suitable for a broad variety of uses.\cite{racket-org} It includes as
built-in packages tools to develop a dynamically-generated website, as
well as for interfacing with a large variety of databases from
different vendors. 

%%%%%%%%%%%%%%%%%%%%%%%%%%%%%%%%%%%%%%%%%%%%%%%%%%%%%%%%%%%%%%%%%%%%
\section{Design}\label{sec:design}
%%%%%%%%%%%%%%%%%%%%%%%%%%%%%%%%%%%%%%%%%%%%%%%%%%%%%%%%%%%%%%%%%%%%
The database itself is described by a schema, which is essentially
some logical design of the database stipulating things such as how the
tables are interrelated, what tables have which attributes, and how
those attributes are constrained. The schema for the table in this
database is as follows:
\newline
\textbf{Donor}
\newline
\begin{tabular}{| l | r |}
  \hline
  Attribute & Constraints \\
  \hline
  id & serial primary key \\
  lastname & varchar(255) not null \\
  firstname & varchar(255) not null \\
  bloodtype & varchar(3) not null \\
  address & varchar(255) \\
  testsdone & text[][] \\
  knowndiseases & text[][] \\
  lastdonationdate & date \\
  phone & char(14) \\
  \hline
\end{tabular}

As a demonstration of the database's capability and real-world
usefulness, I have elected to build a webpage. This webpage permits a
user, here speculated to be bloodbank nurses or doctors, to view
donors, and to manipulate this data accordingly. They would be able to
view details about a patient they select from a brief view, and then
be able to find a list of patients who could donate blood to this
user.
I've chosen Racket because of the aforementioned tight integration of
website devlopment capability and database interaction. These
qualities make Racket an ideal platform for implementing this design.

%%%%%%%%%%%%%%%%%%%%%%%%%%%%%%%%%%%%%%%%%%%%%%%%%%%%%%%%%%%%%%%%%%%%
\section{Analysis}\label{sec:analysis}
%%%%%%%%%%%%%%%%%%%%%%%%%%%%%%%%%%%%%%%%%%%%%%%%%%%%%%%%%%%%%%%%%%%%
<UNFINISHED>
%%%%%%%%%%%%%%%%%%%%%%%%%%%%%%%%%%%%%%%%%%%%%%%%%%%%%%%%%%%%%%%%%%%%
\section{Conclusion}\label{sec:conc}
%%%%%%%%%%%%%%%%%%%%%%%%%%%%%%%%%%%%%%%%%%%%%%%%%%%%%%%%%%%%%%%%%%%%
<UNFINISHED>
%% Winery~\cite{Winery} is graphical modeling tool.

%% \begin{figure}
%% Simple Figure
%% \caption{Simple Figure}
%% \label{fig:simple}
%% \end{figure}

%% \begin{table}
%% \caption{Simple Table}

%% \label{tab:simple}
%% Simple Table
%% \end{table}

%% cref Demonstration: Cref at beginning of sentence, cref in all other cases.

%% \Cref{fig:simple} shows a simple fact, although \cref{fig:simple} could also show something else.

%% \Cref{tab:simple} shows a simple fact, although \cref{tab:simple} could also show something else.

%% \Cref{sec:intro} shows a simple fact, although \cref{sec:intro} could also show something else.

%% Brackets work as designed:
%% <test>

%% The symbol for powerset is now correct: $\powerset$ and not a Weierstrass p ($\wp$).

%% \begin{inparaenum}
%% \item All these items...
%% \item ...appear in one line
%% \item This is enabled by the paralist package.
%% \end{inparaenum}

%% ``something in quotes'' using plain tex or use \enquote{the enquote command}.

%In the bibliography, use \texttt{\textbackslash textsuperscript} for ``st'', ``nd'', ...:
%E.g., \enquote{The 2\textsuperscript{nd} conference on examples}.

%%%%%%%%%%%%%%%%%%%%%%%%%%%%%%%%%%%%%%%%%%%%%%%%%%%%%%%%%%%%%%%%%%%%%%%%%%%%%%%
\bibliographystyle{splncs03}
\bibliography{finalpaper}

%All links were last followed on October 5, 2014.
%%%%%%%%%%%%%%%%%%%%%%%%%%%%%%%%%%%%%%%%%%%%%%%%%%%%%%%%%%%%%%%%%%%%%%%%%%%%%%%

\end{document}

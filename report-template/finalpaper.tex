\documentclass[runningheads,a4paper]{llncs}

\usepackage[american]{babel}

\usepackage{graphicx}

%extended enumerate, such as \begin{compactenum}
\usepackage{paralist}

%put figures inside a text
%\usepackage{picins}
%use
%\piccaptioninside
%\piccaption{...}
%\parpic[r]{\includegraphics ...}
%Text...

%Sorts the citations in the brackets
%\usepackage{cite}

%for easy quotations: \enquote{text}
\usepackage{csquotes}

\usepackage[T1]{fontenc}

%enable margin kerning
\usepackage{microtype}

%better font, similar to the default springer font
%cfr-lm is preferred over lmodern. Reasoning at http://tex.stackexchange.com/a/247543/9075
\usepackage[%
rm={oldstyle=false,proportional=true},%
sf={oldstyle=false,proportional=true},%
tt={oldstyle=false,proportional=true,variable=true},%
qt=false%
]{cfr-lm}
%
%if more space is needed, exchange cfr-lm by mathptmx
%\usepackage{mathptmx}

%for demonstration purposes only
\usepackage[math]{blindtext}

%enable hyperref without colors and without bookmarks
\usepackage[
%pdfauthor={},
%pdfsubject={},
%pdftitle={},
%pdfkeywords={},
bookmarks=false,
breaklinks=true,
colorlinks=true,
linkcolor=black,
citecolor=black,
urlcolor=black,
%pdfstartpage=19,
pdfpagelayout=SinglePage
]{hyperref}
%enables correct jumping to figures when referencing
\usepackage[all]{hypcap}

%enable \cref{...} and \Cref{...} instead of \ref: Type of reference included in the link
\usepackage[capitalise,nameinlink]{cleveref}
%Nice formats for \cref
\crefname{section}{Sect.}{Sect.}
\Crefname{section}{Section}{Sections}
\crefname{figure}{Fig.}{Fig.}
\Crefname{figure}{Figure}{Figures}

\usepackage{xspace}
%\newcommand{\eg}{e.\,g.\xspace}
%\newcommand{\ie}{i.\,e.\xspace}
\newcommand{\eg}{e.\,g.,\ }
\newcommand{\ie}{i.\,e.,\ }

%introduce \powerset - hint by http://matheplanet.com/matheplanet/nuke/html/viewtopic.php?topic=136492&post_id=997377
\DeclareFontFamily{U}{MnSymbolC}{}
\DeclareSymbolFont{MnSyC}{U}{MnSymbolC}{m}{n}
\DeclareFontShape{U}{MnSymbolC}{m}{n}{
    <-6>  MnSymbolC5
   <6-7>  MnSymbolC6
   <7-8>  MnSymbolC7
   <8-9>  MnSymbolC8
   <9-10> MnSymbolC9
  <10-12> MnSymbolC10
  <12->   MnSymbolC12%
}{}
\DeclareMathSymbol{\powerset}{\mathord}{MnSyC}{180}

%improve wrapping of URLs - hint by http://tex.stackexchange.com/a/10419/9075
\makeatletter
\g@addto@macro{\UrlBreaks}{\UrlOrds}
\makeatother

% correct bad hyphenation here
\hyphenation{op-tical net-works semi-conduc-tor}

\begin{document}

%Works on MiKTeX only
%hint by http://goemonx.blogspot.de/2012/01/pdflatex-ligaturen-und-copynpaste.html
%also http://tex.stackexchange.com/questions/4397/make-ligatures-in-linux-libertine-copyable-and-searchable
%This allows a copy'n'paste of the text from the paper
\input glyphtounicode.tex
\pdfgentounicode=1

\title{Building a Bloodbank Database}
%If Title is too long, use \titlerunning
%\titlerunning{Short Title}

%Single insitute
\author{Nicholas Gordon}
%If there are too many authors, use \authorrunning
%\authorrunning{First Author et al.}
\institute{University of Memphis}

%Multiple insitutes
%Currently disabled
%
\iffalse
%Multiple institutes are typeset as follows:
\author{Firstname Lastname\inst{1} \and Firstname Lastname\inst{2} }
%If there are too many authors, use \authorrunning
%\authorrunning{First Author et al.}

\institute{
Insitute 1\\
\email{...}\and
Insitute 2\\
\email{...}
}
\fi
			
\maketitle

\begin{abstract}
  A report on the appropriateness, development, and construction of a
  database suitable for use by a blood bank. Includes short
  introduction to databases and competing vendors and web technologies
  as well. Discussed are the design decisions made, including a
  discussion of the schema used for the database, as well as the user
  interface design. Finally, a discussion of the implementation itself
  is given, including problems faced, analysis of the solution, and
  future expansions of the solution.
\end{abstract}

%% \keywords{...}

%%%%%%%%%%%%%%%%%%%%%%%%%%%%%%%%%%%%%%%%%%%%%%%%%%%%%%%%%%%%%%%%%%%%%%%%%%%%%%%
\section{Introduction}\label{sec:intro}
%%%%%%%%%%%%%%%%%%%%%%%%%%%%%%%%%%%%%%%%%%%%%%%%%%%%%%%%%%%%%%%%%%%%%%%%%%%%%%%
Databases are ubiquitous in computers. They are an extremely powerful
tool for organizing data, which is an imperative in computer science;
randomness is tantamount to uselessness. Once data is organized, it
can be manipulated in powerful ways to derive meaning from data. One
useful example of this is a blood bank. In a blood bank, the data are
a pool of patients, each with varying blood types, blood-borne
diseases, names, addresses, etc. Without organization, the best you
could hope to do is keep track of everyone. With organization, you can
generate lists of patients to contact for donation, who can donate to
some arbitrary patient, or even location-based demographics on
frequency of donation, disease distribution, etc.

Now we come to the database itself. We could simply organize these
lists by hand, periodically generating all of these metrics, but we
can use computers to muscle this data into formats we perfer for
us. This is the impetus for this project: generate a database that can
be populated with data from a blood bank which will provide to the
bank useful meaning from their data. The rest of this report will deal
with some knowledge upon which this report is incumbent, such as
differing database vendors and web technologies available for
interfacing with these databases, the overall design of the database
that has been produced, including the structure of the tables and the
database itself, and a discussion of the problems faced, the solution
presented, and its appropriateness to the problem at hand.

%%%%%%%%%%%%%%%%%%%%%%%%%%%%%%%%%%%%%%%%%%%%%%%%%%%%%%%%%%%%%%%%%%%%
\section{Related Work}\label{sec:related}
%%%%%%%%%%%%%%%%%%%%%%%%%%%%%%%%%%%%%%%%%%%%%%%%%%%%%%%%%%%%%%%%%%%%
In order to fully understand the decisions made during this project
and the design of the database given, some background knowledge needs
to be established. Given that a not-narrow audience is likely to be
reading a paper such as this, this explanation is warranted. Databases
often are just referred to as ``the database'' because of their
ubiquity, but the truth of the matter is that there are a great many
databases that any sysadmin must choose from, each having their own
quirks, costs, and intended uses. According to db-engines.com, which
uses a variety of metrics to determine database popularity, the five
most popular databases in February 2016\cite{db-engine} are, in order:
\begin{enumerate}
\item Oracle
\item MySQL
\item Microsoft SQL Server
\item MongoDB
\item PostgreSQL
\end{enumerate}
This list is not linear, however. The top three entries have
reasonably close numbers in terms of popularity, but MongoDB and
PostgreSQL have only about 30\% each of the share of Microsoft SQL
Server. Most databases are relational, which means their records are
stored in tables or similar, and relate multiple fields together in
items called records. These databases are the dominant paradigm, and
only MongoDB on this list is not a relational database. MongoDB fits
into another paradigm known as document-store database, which
effectively are a dictionary that stores complex objects in a
key-value system. The differences between these databases largely come
down to individual feature support, cost of use, and
enterprise-support. Of note is that MySQL, MongoDB, and PostgreSQL are
all open-source software and as such are free to use or have a free
version available.

Racket is a Scheme derivative, which emphasizes being a platform for
extending the language and being a full-spectrum programming language
suitable for a broad variety of uses.\cite{racket-org} It includes as
built-in packages tools to develop a dynamically-generated website, as
well as for interfacing with a large variety of databases from
different vendors.

With regards to the bloodbank itself, when determining something as
straightforwarded as who can donate blood to who, we have an important
classifier to understand: the ABO blood group system and Rh
factor. These two qualifiers decide what kind of blood any given
person can receive.\cite{redcross-types} This is critically important
to the process; if you give a patient blood they cannot receive, their
body could incorrectly identify the transfused blood cells as foreign
invaders, and as a result their immune system could have a major
reaction, causing severe bodily danger, such as shock or kidney
failure.\cite{nih-blood}. As a result, there is a strong motivation to
get the relevant information correct.

Additionally, there exist infections that reside in the blood of the
infected. Many of these infections can be transmitted through the
blood, including dengue fever, the hepatitus viruses, HIV, and malaria
among others.\cite{cdc-diseases} As such, donations must be screened
before they can be given to a needful recipient to avoid causing
transmission of these infections. Further, a record of which donors
are known to have certain diseases can help a bloodbank prioritize
which people they contact to request donations, or provide useful
information to viral epidemiology studies.

%%%%%%%%%%%%%%%%%%%%%%%%%%%%%%%%%%%%%%%%%%%%%%%%%%%%%%%%%%%%%%%%%%%%
\section{Design}\label{sec:design}
%%%%%%%%%%%%%%%%%%%%%%%%%%%%%%%%%%%%%%%%%%%%%%%%%%%%%%%%%%%%%%%%%%%%
The database itself is described by a schema, which is essentially
some logical design of the database stipulating things such as how the
tables are interrelated, what tables have which attributes, and how
those attributes are constrained. The schema for the table in this
database is as follows:
\newline
\textbf{Donor}
\newline
\begin{tabular}{| l | r |}
  \hline
  Attribute & Constraints \\
  \hline
  id & serial primary key \\
  lastname & varchar(255) not null \\
  firstname & varchar(255) not null \\
  bloodtype & varchar(3) not null \\
  address & varchar(255) \\
  testsdone & text[] \\
  knowndiseases & text[] \\
  lastdonationdate & date \\
  phone & char(14) \\
  \hline
\end{tabular}

As a demonstration of the database's capability and real-world
usefulness, a webpage was built. This webpage permits a user, here
speculated to be bloodbank nurses or doctors, to view donors, and to
manipulate this data accordingly. They are be able to view details
about a patient they select from a brief view, and then find a list of
patients who could donate blood to this user. Naturally, such a
webpage should also enable the user to add, remove, and modify the
patients in the database. PostgreSQL was chosen as the backend
database system to implement the solution on.

The web-based application design that was chosen is structured as
follows: the builk of the application is hosted on the initial site: a
search form to narrow results down, and the related list of donors
beneath that. For editing donors, viewing more donor details, and
finding a list of compatible donors, you are taken to new pages for
each of these things. Hosted on these pages are the relevant forms
listed in HTML tables, and the content-modifying actions (edit,
delete, add) all have additional confirmation pages before any true
action is committed.

%%%%%%%%%%%%%%%%%%%%%%%%%%%%%%%%%%%%%%%%%%%%%%%%%%%%%%%%%%%%%%%%%%%%
\section{Analysis}\label{sec:analysis}
%%%%%%%%%%%%%%%%%%%%%%%%%%%%%%%%%%%%%%%%%%%%%%%%%%%%%%%%%%%%%%%%%%%%
My experiences and encounters with this project have been varied and
interesting. The biggest stumbling block has by far been the
integration of the database with the developed frontend website. Both
design choices and implementation issues will be discussed, including
schema design, variable typing, data validation, and the peculiarities
of Racket in general.

Racket was chosen due to the language having built-in packages for
both web-page generation and serving and database interaction. As a
result, there was no required code to transform the data between
systems or to act as 'duct tape'. This makes the system simpler to
maintain, and means that the required skill level to maintain the code
is lower than it would otherwise be, using say using some typical
combination as LAMP, which requires four separate skill-sets. The
Racket/PostgreSQL stac used here requires only two. Additionally,
Racket was a good choice because of my prior familiarity with it,
avoiding the need to inves time in learning a new language.

I correctly expected addition and updating to be more challenging than
the other features, owing to considerations like data validation and
integrity requirements. Because of the breadth of information
regarding a blood bank, this was left mostly-undeveloped; a future
iteration of this could include this.

I expected addition and updating to be more
challenging, owing to data validation requirements.

  I've chosen
Racket because of the aforementioned tight integration of website
devlopment capability and database interaction. These qualities make
Racket an ideal platform for implementing this design. 

First, decisions about what fields to include in the 'donor/patient'
model had to be made. Ultimately, the choice was to make the model as
broad as possible, hoping to cover as many use-cases as possible; this
is the justification for making only first name, last name and
bloodtype requirements. This selection of constraints covers the
greatest cross-section of possible donors. Further, the decision to
limit retained attributes to simple demographics and medical
information is practical. A division of information exists between
clerical information, such as billing, weight requirements, blood
quality, reimbursement, etc., and less ephemeral, qualitative data
like name, blood type, address. While a database could be used
similarly to retain and manage the clerical information, the solution
presented here was intended to be a broadly-applicable one, and so an
omission of complexity was valid.

Second, a decision had to be made about how to store the list of tests
that have been done on the donor's blood, and what diseases a given
donor is known to have. Due to the nature of arrays, the traditional
SQL approach to arrays is to instead use additional tables and join
those tables as a foreign key. The motivations for eschewing this and
using Postgres's built-in, declarative-like array structures are
twofold: it allows for more intuitive abstraction about the data, and
it simplifies the work needed to convert the data gathered in the
frontend to the backend database. Additionally, because the number of
entries in these arrays will likely be small (<10), any performance
advantage that would be gained from the traditional approach is
negligibly small.

After the implementation of this system, it has become clear that none
of the features showcased are vendor-dependent; the database backend
for this is quite flexible, pending only some code tweaks to the
database interfacing code. This is not a huge revelation considering
the scope and complexity of the web app.
%%%%%%%%%%%%%%%%%%%%%%%%%%%%%%%%%%%%%%%%%%%%%%%%%%%%%%%%%%%%%%%%%%%%
\section{Future Works}\label{sec:fworks}
%%%%%%%%%%%%%%%%%%%%%%%%%%%%%%%%%%%%%%%%%%%%%%%%%%%%%%%%%%%%%%%%%%%%
While this web application is a working system, it clearly lacks many features and lots of 'polish'. Due to time constraints I have not been able to add some features in, and future development could yield improvements. Several of the most important features are listed here:
\begin{itemize}
\item The aesthetics of the page could be improved by use of CSS with appropriate re-design of the underlying HTML.
\item An interface redesign could also provide cleaner results, using bigger, modern buttons in place of the standard HTML form buttons.
\item The current system for updating and adding patients provides no systematic validation for what blood diseases and tests are permitted; an improved system would be to maintain a whitelist for each set and check inputs against them.
\item Calendar functionality would be useful, with regards to scheduling people for donations. Such a system might permit, for example, bringing up a list of donors scheduled for a day, so a worker could call and remind them of an appointment.
\end{itemize}

It is also not difficult to see how a system like this could be monetized. In a world where HIPAA places draconian requirements on the currency of software used to process and store patient data, software-as-a-service (SAAS) is becoming more popular, as companies seek to reduce their number of technology headaches. As this is built on a web platform, it is on a clear track to SAAS.
%%%%%%%%%%%%%%%%%%%%%%%%%%%%%%%%%%%%%%%%%%%%%%%%%%%%%%%%%%%%%%%%%%%%
\section{Conclusion}\label{sec:conc}
%%%%%%%%%%%%%%%%%%%%%%%%%%%%%%%%%%%%%%%%%%%%%%%%%%%%%%%%%%%%%%%%%%%%
In conclusion, it is easy to see why a blood bank would need such a database. The test database was seeded with 1000 records, and response time was nearly-instant on a consumer-grade laptop. Compare that to a paper storage system, where information fetch times become the biggest bottleneck to effective operation, and not logistics. Regarding the technical implementation, this project was a lesson in how quickly a piece of software can grow in complexity; from simple create-read-update-delete the project grew to specific reads, data-type validation, constrained views, and an interesting exercise in user experience: do you force two clicks to do a content-modifying action, or not? Do you present all the listings at once, or have a 'more' button? Regardless, a system such as the one presented here is not technically demanding, and in fact very few good 'how-to's exist for this sort of Racket project; this was mostly free-handed. However, a lack of technical complexity is not to discount interface complexity or use-case complexity. This project illuminated many of the issues associated with end-user software, like sandboxing and security, too.

%%%%%%%%%%%%%%%%%%%%%%%%%%%%%%%%%%%%%%%%%%%%%%%%%%%%%%%%%%%%%%%%%%%%%%%%%%%%%%%
% BIBLIOGRAPHY
%%%%%%%%%%%%%%%%%%%%%%%%%%%%%%%%%%%%%%%%%%%%%%%%%%%%%%%%%%%%%%%%%%%%%%%%%%%%%%%
\bibliographystyle{splncs03}
\bibliography{finalpaper}
%%%%%%%%%%%%%%%%%%%%%%%%%%%%%%%%%%%%%%%%%%%%%%%%%%%%%%%%%%%%%%%%%%%%%%%%%%%%%%%

\end{document}
